\documentclass[a4paper]{article}
%\usepackage{fullpage}
\usepackage[bottom=1in, left=1in, top=1in, right=2in]{geometry}
\usepackage[utf8]{inputenc}
\usepackage{graphicx}
\usepackage{amsmath}
\usepackage{amssymb}
\usepackage{amsthm}
\usepackage{algorithm2e}
\usepackage{listings}
\pagestyle{myheadings}
\markright{}

\begin{document}
\title{6.945 Final Project\\
Tscheme: Using type inference to automatically complain about the programmer's code}
\date{\today}
\author{Aaron Graham-Horowitz, Ben Zinberg, Jan Polášek}
\maketitle

\newpage

\section{Introduction}

Tscheme is a static code analyser over a subset of MIT Scheme
that aims to reduce the number of bugs in user's code.
It uses static type analysis to derive supersets of values that can be taken on by
(sub)expressions.
If an empty set is derived, a type error is bound to happen and the error is reported.
If the derived set of values is non-empty, the user is allowed to query the sets of
values expressions can take on to get more information about the code they have written.

There are two common ways of doing static analysis.
One approach is static typing, which imposes type rules which are
over-approximations of what can happen in correct programs.
A good static typing scheme will provide some sort of type safety,
guarantees that will be enforced and that the programmer can rely on.
It is for example possible to statically determine that an integer plus
operator will never be ca

\end{document}
