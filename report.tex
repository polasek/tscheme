\documentclass[a4paper]{article}
%\usepackage{fullpage}
\usepackage[bottom=1in, left=1in, top=1in, right=2in]{geometry}
\usepackage[utf8]{inputenc}
\usepackage{graphicx}
\usepackage{amsmath}
\usepackage{amssymb}
\usepackage{amsthm}
\usepackage{listings}
\pagestyle{myheadings}
\markright{}

\begin{document}
\title{6.945 Final Project\\
Tscheme: Using type inference to automatically complain about the programmer's code}
\date{\today}
\author{Aaron Graham-Horowitz, Ben Zinberg, Jan Polášek}
\maketitle

\newpage

\section{Introduction}

Tscheme is a static code analyser over a subset of MIT Scheme
that aims to reduce the number of bugs in user's code.
It uses static type analysis to derive supersets of values that can be taken on by
(sub)expressions.
If an empty set is derived, a type error is bound to happen and the error is reported.
If the derived set of values is non-empty, the user is allowed to query the sets of
values expressions can take on to get more information about the code they have written.

There are two common ways of doing static analysis.
One approach is static typing, which imposes type rules which are
over-approximations of what can happen in correct programs.
A good static typing scheme will provide some sort of type safety -
guarantees that will be enforced and that the programmer can rely on.
It is for example possible to statically determine that an integer ``plus''
operator will never be called with a string passed as an argument.
The disadvantage of any reasonable type system over a sufficiently powerful language
is that there will always be valid programs that do not violate the notion of
type safety guaranteed by the type system that nevertheless will not type check.
In other words, the type checker will never produce false-positives,
but can produce false-negatives.
As we wanted to have a static analyser that works on an existing Scheme
language (or a subset of), we decided to go in the other direction,
which is a code analyser.
Unlike a type checker, a code analyser will never produce false-negatives,
but can produce false-positive results.

\section{Assumptions}

Tscheme works on a functional subset of MIT Scheme.  Mutation greatly limits 
our ability to make inferences about types, as any variable that might be 
accessible outside the current scope could potentially be changed to any type 
at almost any time.  Rather than work around this complication we require that
submitted code be free of side-effects.  We make a few other restrictions on 
the code.  We do not allow for procedures with variable number of arguments;
this makes it easier to reason about the types of functions.  We do not, by 
default, support all Scheme primitive procedures, however each scheme 
primitive that we handle is explicitly given a type in our code, and more can 
easily be added as needed.  In a production version of Tscheme we would 
include all MIT Scheme primitives, but for this early version we include only 
some of most commonly used primitives for testing purposes.

Our implementation of type analysis is strict in the requirement of not producing 
false negatives.  To maintain this strictness we do not collect constraints 
from within the branches of a conditional (we do get constraints from the 
predicate, however).  Because we cannot determine which branches will be 
accessible, we assume the return value of an {\texttt if} or {\texttt cond}
statement is {\texttt *top*}, the set of all possible types.

%Overview of the subset of scheme we handle, what things do not work with our
%analysis (ifs) and why

\section{Workflow}

%The general workflow, i.e. generating constraints, applying them to a mapping table
Our program is split into two independent parts.
First, we have a procedure that takes in code and generates ``type'' constraints.
These restrict the values that can be taken on by subexpressions of the
code that is being analysed.
We generate type variables for both variables and expressions
and use them to create the right restrictions by using constraints.

Next, we have a procedure that takes a list of constraints and uses them to
refine the sets of values that a type variable (representing either a subespression
or a code variable) can take on.
Moreover, it makes sure that no type variable 

\section{Data structures}

%Constraints and types

\section{Constraint generation}

\section{Refining types using constraints}

\section{Development process}

%work division, description of the general process

\section{Discussion and future work}

%limitations of our approach, what can be done about it
%directions in which this can be improved, e.g. emacs integration

\end{document}
